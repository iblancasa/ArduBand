\chapter*{}
%\thispagestyle{empty}
%\cleardoublepage

%\thispagestyle{empty}

\begin{titlepage}


\setlength{\centeroffset}{-0.5\oddsidemargin}
\addtolength{\centeroffset}{0.5\evensidemargin}
\thispagestyle{empty}

\noindent\hspace*{\centeroffset}\begin{minipage}{\textwidth}

\centering

\includegraphics{imagenes/logo.png}
 \vspace{0.5cm}


{\Huge\bfseries ArduBand\\
}
\noindent\rule[-1ex]{\textwidth}{3pt}\\[3.5ex]
{\large\bfseries Vestible inalámbrico para coordinación de músicos en una banda\\[4cm]}
\end{minipage}

\vspace{3cm}
\noindent\hspace*{\centeroffset}\begin{minipage}{\textwidth}
\centering

\textbf{Autor}\\ {Israel Blancas Álvarez}\\[2.5ex]
\textbf{Director}\\
{Samuel Francisco Romero García}\\[2.5cm]
\includegraphics[width=0.15\textwidth]{imagenes/atc.png}\\[0.1cm]
\textsc{Departamento Arquitectura y Tecnología de Computadores}\\
\textsc{---}\\
\end{minipage}
\addtolength{\textwidth}{\centeroffset}
\vspace{\stretch{2}}


\end{titlepage}



\cleardoublepage
\thispagestyle{empty}

\begin{center}
{\large\bfseries ArduBand: sistema  \textit{wearable} para sincronización de bandas de música}\\
\end{center}
\begin{center}
Israel Blancas Álvarez\\
\end{center}

%\vspace{0.7cm}
\noindent{\textbf{Palabras clave}: ZigBee, Metrónomo, Arduino, WSN, XBee, Android}\\

\vspace{0.7cm}
\noindent{\textbf{Resumen}}\\
En el presente trabajo, el lector podrá encontrar cómo se ha desarrollado un sistema
\textit{wearable} que, mediante vibración, facilita a los intérpretes de una banda de
música seguir el mismo \textit{tempo}. De esta forma, se ayudará a que todos los miembros del conjunto
sigan el mismo pulso y puedan mantenerlo constante durante toda la ejecución.\\

Esta necesidad ha despertado el interés de algunas compañías, que han desarrollado sistemas como
llamado ``Body Beat"\ (de la marca ``Peterson") que, a pesar de su gran abanico de funciones,
no ha tenido el éxito esperado en el mercado al tener un coste elevado. Un sistema que hiciese las
veces de metrónomo wireless (característica principal y más atrayente del artículo antes mencionado)
a un precio menor, atraería a muchos más clientes.\\

Para que todos los integrantes de la agrupación puedan llevar el mismo pulso, se ha creado una red
inalámbrica de sensores (\textit{WSN}) que permite la sincronización de los aparatos que portan todos los músicos.
La red de sensores ha sido concebida utilizando la implementación de ZigBee propuesta por la empresa “Digi
International”, mientras la lógica de los circuitos se ha puesto en manos de la plataforma hardware Arduino
(utilizándose diversas versiones del mismo).\\

El tipo de red que forman los dispositivos ``XBee ZigBee"\ es de tipo malla pero, utilizando la configuración
de las motas, se ha pasado a tener una topología de estrella. Dispone de dos tipos de  \textit{wearables}:
  \begin{itemize}
  \item Director: su mota juega el papel de ``coordinador" de la red. Es quien crea la red y establece los caminos que
  deben seguir las comunicaciones que haya en la red. Además de la comunicación con los otros elementos del sistema,
  tiene la opción de conectar, a través de Bluetooth, con un dispositivo Android.
  \item Músico: contiene una mota del tipo ``dispositivo final". Simplemente, recibe (del dispositivo ``director")
  el \textit{tempo} con el que el micromotor vibrador debe activarse (además, al llegar el paquete con la traza de datos, se
  sincroniza con el resto de dispositivos de la red, de forma que todos vibren a la vez).
  \end{itemize}

Como se comentaba cuando se hablaba del dispositivo del director, se puede conectar
con un dispositivo móvil Android. Es necesario que el director indique el tempo que el
sistema debe marcar a los músicos, esta aplicación móvil es la que transfiere al controlador la
velocidad a la que debe compaginar a los intérpretes (para evitar sobrecargar la red, se hace una
coordinación cada cierto tiempo -por si hubiera habido retrasos en la organización inicial- y cada
dispositivo, en función del \textit{tempo} que se le ha enviado, mantiene el pulso). Buscando facilitar al usuario la utilización
de esta tecnología, se ha desarrollado tanto una aplicación para teléfonos móviles.\\

Una vez creada la base del sistema, es posible crear nuevas funcionalidades (como la instalación de un
sensor de vibración que permita a los percusionistas obtener retroalimentación para saber si los intérpretes
están ejecutando la partitura siguiendo el tiempo marcado o la posibilidad de pasar lista -conociendo qué sensores
se encuentran activos en la red-).\\

\cleardoublepage


\thispagestyle{empty}


\begin{center}
{\large\bfseries ArduBand: wearable system to synchronize music bands}\\
\end{center}
\begin{center}
Israel Blancas\\
\end{center}

%\vspace{0.7cm}
\noindent{\textbf{Keywords}: ZigBee, Metronome, Arduino, WSN, XBee, Android}\\

\vspace{0.7cm}
\noindent{\textbf{Abstract}}\\

One of the principles of engineering is creating solutions to the problems of users.
In this dissertation, the reader will be able to find how a wearable system has been developed
helping musicians to go on the same “tempo” through vibration while they are playing music.
This device sends the same pulse to all musicians and keep the “tempo” constant.\\

Some companies have developed similar systems. For example, Peterson created an item called “Body Beat”.
It has a big range of functions and odds but it is too expensive for the majority of musicians. One device
cheaper than it only with synchronization functions (the most important ability of this system) could attract
more buyers. Furthermore, to create a free and open hardware platform could interest other developers to improve
the functionality of this product (helping musicians, music teachers and other music professionals to play music
with a better quality).\\

All music band’s components need to have the same pulse (if each instrumentalist had a different pulse, each
one would read his score in a different speed and it would be a problem). It is possible because it has been
created using a wireless sensor network (they enable communication with a very low energy cost). Then, when all
nodes are synchronized, they know when they have to start the vibrations. But they do not have to be vibrating all
time. They have to vibrate constantly as many times as the director said (for example, if compasses are of 4/4 and
tempo 60 bpm -Beats Per Minute-, each node will vibrate 1 time per second).\\

This network has been deployed using ZigBee implementation of the “Digi International” company. Also, circuit logic
has been put into operation using the hardware platform called Arduino (various versions of this, like Arduino Lilypad,
Arduino Uno or Arduino Leonardo).\\

In the following pages , it is explained in more detail why certain decisions have been taken (some experiments in
time between node's communications, explanations about XBee communication packets...).\\

XBee Zigbee’s devices are organized in mesh network but, changing each node’s configuration, we have now a network with
star topology. There are two kinds of wearables:
  \begin{itemize}
  \item Music director: it is the coordinator of network. It is who start the network and establish the paths of communication packets between all nodes. In addition, it is able to send data to every node and receive data from an Android device via Bluetooth. Only one in each network.
  \item Musician: it is composed of an “end device” node. It receives (from “music director device”) data (which contains “tempo”). Arduino takes this tempo and performs calculations to decide when it has to activate or deactivate a vibration motor (this motor helps musician to keep track of the correct pulse). There is one for each musician (in network, there will be as nodes as musicians).
  \end{itemize}


Director’s device can be connected to an Android device (which could be a smartphone, a tablet or a smartwatch, for example).
This application allows director to indicate music’s
``tempo".\\

Taking this base, it is possible to create new functions such as installing a vibration sensors in drums to measure the
tempo of the band and provide a feedback to the director. Another possibility could be rollcalling at the band (only it
is necessary show what nodes are in the network at the moment).\\


\chapter*{}
\thispagestyle{empty}

\noindent\rule[-1ex]{\textwidth}{2pt}\\[4.5ex]

Yo, \textbf Israel Blancas Álvarez, alumno de la titulación Grado de Ingeniería Informática de la \textbf{Escuela Técnica Superior
de Ingenierías Informática y de Telecomunicación de la Universidad de Granada}, con DNI XXXXXXXXX, autorizo la
ubicación de la siguiente copia de mi Trabajo Fin de Grado en la biblioteca del centro para que pueda ser
consultada por las personas que lo deseen.

\vspace{6cm}

\noindent Fdo: Israel Blancas Álvarez

\vspace{2cm}

\begin{flushright}
Granada a 6 de julio de 2015.
\end{flushright}


\chapter*{}
\thispagestyle{empty}

\noindent\rule[-1ex]{\textwidth}{2pt}\\[4.5ex]

D. \textbf{Samuel Francisco Romero García}, Profesor del Área de Arquitectura y Tecnología de Computadores del Departamento Arquitectura y Tecnología de Computadores de la Universidad de Granada.

\vspace{0.5cm}


\vspace{0.5cm}

\textbf{Informa:}

\vspace{0.5cm}

Que el presente trabajo, titulado \textit{\textbf{ArduBand, vestible inalámbrico para coordinación de músicos en una banda}},
ha sido realizado bajo su supervisión por \textbf{Israel Blancas Álvarez}, y autorizo la defensa de dicho trabajo ante el tribunal
que corresponda.

\vspace{0.5cm}

Y para que conste, expiden y firman el presente informe en Granada a 6 de julio de 2015.

\vspace{1cm}

\textbf{El tutor:}

\vspace{5cm}

\noindent \textbf{Samuel Francisco Romero García}

\chapter*{Agradecimientos}
\thispagestyle{empty}

       \vspace{1cm}

A aquellos maestros y maestras del Colegio Vicente Aleixandre de Granada que siempre creyeron en mí.\\ \\ \\

A los profesores y profesoras del Instituto Mariana Pineda de Granada que me motivaron para
dar cada día lo mejor de mí.\\ \\ \\

A los profesores y profesoras de la ETSIIT que supieron enseñarme a ser más creativo y a proponerme a mí mismo nuevos retos.\\ \\ \\

A mis compañeros de clase por tantas horas de trabajo y momentos, buenos y malos, compartidos.\\ \\ \\

A mis amigos, por que han estado siempre ahí para animarme en los malos momentos.\\ \\ \\

A mi familia, por que, sin su dedicación durante todos estos años, su apoyo en los momentos
de duda y su compresión, no habría llegado a nada.
