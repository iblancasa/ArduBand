\documentclass[a4paper,11pt]{book}
\usepackage{listings}
\usepackage[utf8]{inputenc}
\usepackage[spanish]{babel}
\usepackage{eurosym}
\usepackage[table,xcdraw]{xcolor}
\usepackage{amssymb}
\usepackage{amsmath}
\usepackage{tikz}
\usepackage{caption}
\usepackage{algorithm,algpseudocode}


\decimalpoint
\usepackage{dcolumn}
\newcolumntype{.}{D{.}{\esperiod}{-1}}
\makeatletter
\addto\shorthandsspanish{\let\esperiod\es@period@code}
\makeatother


%\usepackage[chapter]{algorithm}
\RequirePackage{verbatim}
%\RequirePackage[Glenn]{fncychap}
\usepackage{fancyhdr}
\usepackage{graphicx}
\usepackage{afterpage}

\usepackage{longtable}

\usepackage[pdfborder={000}]{hyperref} %referencia



\newcommand{\myTitle}{ArduBand\xspace}
\newcommand{\myDegree}{Grado en Ingeniería Informática\xspace}
\newcommand{\myName}{Israel Blancas Álvarez\xspace}
\newcommand{\myProf}{Samuel Francisco Romero García\xspace}
\newcommand{\myFaculty}{Escuela Técnica Superior de Ingenierías Informática y de
Telecomunicación\xspace}
\newcommand{\myFacultyShort}{E.T.S. de Ingenierías Informática y de
Telecomunicación\xspace}
\newcommand{\myDepartment}{Departamento de Arquitectura y Tecnología de Computadores\xspace}
\newcommand{\myUni}{\protect{Universidad de Granada}\xspace}
\newcommand{\myLocation}{Granada\xspace}
\newcommand{\myTime}{\today\xspace}
\newcommand{\myVersion}{Version 0.1\xspace}


\hypersetup{
pdfauthor = {\myName Israel Blancas Álvarez},
pdftitle = {\myTitle ArduBand},
pdfsubject = {},
pdfkeywords = {ZigBee, Metrónomo, Arduino, WSN, XBee, Android},
pdfcreator = {LaTex},
pdfproducer = {pdflatex}
}

\hyphenation{}

\usepackage{url}
\usepackage{colortbl,longtable}
\usepackage[stable]{footmisc}

\addto\captionsspanish{
\def\tablename{Tabla}
}

\tikzset{
every picture/.append style={
  execute at begin picture={\deactivatequoting},
  execute at end picture={\activatequoting}
  }
}


\pagestyle{fancy}
\fancyhf{}
\fancyhead[LO]{\leftmark}
\fancyhead[RE]{\rightmark}
\fancyhead[RO,LE]{\textbf{\thepage}}
\renewcommand{\chaptermark}[1]{\markboth{\textbf{#1}}{}}
\renewcommand{\sectionmark}[1]{\markright{\textbf{\thesection. #1}}}

\setlength{\headheight}{1.5\headheight}

\newcommand{\HRule}{\rule{\linewidth}{0.5mm}}
%Definimos los tipos teorema, ejemplo y definición podremos usar estos tipos
%simplemente poniendo \begin{teorema} \end{teorema} ...
\newtheorem{teorema}{Teorema}[chapter]
\newtheorem{ejemplo}{Ejemplo}[chapter]


\definecolor{gray97}{gray}{.97}
\definecolor{gray75}{gray}{.75}
\definecolor{gray45}{gray}{.45}
\definecolor{gray30}{gray}{.94}


\definecolor{blueS}{RGB}{32, 114, 172}
\definecolor{rojoOscuroS}{RGB}{170, 48, 71}
\definecolor{naranjaS}{RGB}{255, 165, 29}
\definecolor{verdeS}{RGB}{111, 142, 126}
\definecolor{pistachoS}{RGB}{190, 204, 31}

\lstset{ frame=Ltb,
     framerule=0.5pt,
     aboveskip=0.5cm,
     framextopmargin=3pt,
     framexbottommargin=3pt,
     framexleftmargin=0.1cm,
     framesep=0pt,
     rulesep=.4pt,
     backgroundcolor=\color{gray97},
     rulesepcolor=\color{black},
     %
     stringstyle=\ttfamily,
     showstringspaces = false,
     basicstyle=\scriptsize\ttfamily,
     commentstyle=\color{gray45},
     keywordstyle=\bfseries,
     %
     numbers=left,
     numbersep=6pt,
     numberstyle=\tiny,
     numberfirstline = false,
     breaklines=true,
   }

\lstnewenvironment{listing}[1][]
   {\lstset{#1}\pagebreak[0]}{\pagebreak[0]}

\lstdefinestyle{CodigoC}
   {
	basicstyle=\scriptsize,
	frame=single,
	language=C,
	numbers=left
   }

\lstdefinestyle{Consola}
   {basicstyle=\scriptsize\bf\ttfamily,
    backgroundcolor=\color{gray30},
    frame=single,
    numbers=none
   }


\newcommand{\bigrule}{\titlerule[0.5mm]}


%Para conseguir que en las páginas en blanco no ponga cabeceras
\makeatletter
\def\clearpage{%
  \ifvmode
    \ifnum \@dbltopnum =\m@ne
      \ifdim \pagetotal <\topskip
        \hbox{}
      \fi
    \fi
  \fi
  \newpage
  \thispagestyle{empty}
  \write\m@ne{}
  \vbox{}
  \penalty -\@Mi
}
\makeatother

\usepackage{pdfpages}
\begin{document}
\input{portada/portada}
\chapter*{}
%\thispagestyle{empty}
%\cleardoublepage

%\thispagestyle{empty}

\begin{titlepage}


\setlength{\centeroffset}{-0.5\oddsidemargin}
\addtolength{\centeroffset}{0.5\evensidemargin}
\thispagestyle{empty}

\noindent\hspace*{\centeroffset}\begin{minipage}{\textwidth}

\centering

\includegraphics{imagenes/logo.png}
 \vspace{0.5cm}


{\Huge\bfseries ArduBand\\
}
\noindent\rule[-1ex]{\textwidth}{3pt}\\[3.5ex]
{\large\bfseries Vestible inalámbrico para coordinación de músicos en una banda\\[4cm]}
\end{minipage}

\vspace{3cm}
\noindent\hspace*{\centeroffset}\begin{minipage}{\textwidth}
\centering

\textbf{Autor}\\ {Israel Blancas Álvarez}\\[2.5ex]
\textbf{Director}\\
{Samuel Francisco Romero García}\\[2.5cm]
\includegraphics[width=0.15\textwidth]{imagenes/atc.png}\\[0.1cm]
\textsc{Departamento Arquitectura y Tecnología de Computadores}\\
\textsc{---}\\
\end{minipage}
\addtolength{\textwidth}{\centeroffset}
\vspace{\stretch{2}}


\end{titlepage}



\cleardoublepage
\thispagestyle{empty}

\begin{center}
{\large\bfseries ArduBand: sistema  \textit{wearable} para sincronización de bandas de música}\\
\end{center}
\begin{center}
Israel Blancas Álvarez\\
\end{center}

%\vspace{0.7cm}
\noindent{\textbf{Palabras clave}: ZigBee, Metrónomo, Arduino, WSN, XBee, Android}\\

\vspace{0.7cm}
\noindent{\textbf{Resumen}}\\
En el presente trabajo, el lector podrá encontrar cómo se ha desarrollado un sistema
\textit{wearable} que, mediante vibración, facilita a los intérpretes de una banda de
música seguir el mismo \textit{tempo}. De esta forma, se ayudará a que todos los miembros del conjunto
sigan el mismo pulso y puedan mantenerlo constante durante toda la ejecución.\\

Esta necesidad ha despertado el interés de algunas compañías, que han desarrollado sistemas como
llamado ``Body Beat"\ (de la marca ``Peterson") que, a pesar de su gran abanico de funciones,
no ha tenido el éxito esperado en el mercado al tener un coste elevado. Un sistema que hiciese las
veces de metrónomo wireless (característica principal y más atrayente del artículo antes mencionado)
a un precio menor, atraería a muchos más clientes.\\

Para que todos los integrantes de la agrupación puedan llevar el mismo pulso, se ha creado una red
inalámbrica de sensores (\textit{WSN}) que permite la sincronización de los aparatos que portan todos los músicos.
La red de sensores ha sido concebida utilizando la implementación de ZigBee propuesta por la empresa “Digi
International”, mientras la lógica de los circuitos se ha puesto en manos de la plataforma hardware Arduino
(utilizándose diversas versiones del mismo).\\

El tipo de red que forman los dispositivos ``XBee ZigBee"\ es de tipo malla pero, utilizando la configuración
de las motas, se ha pasado a tener una topología de estrella. Dispone de dos tipos de  \textit{wearables}:
  \begin{itemize}
  \item Director: su mota juega el papel de ``coordinador" de la red. Es quien crea la red y establece los caminos que
  deben seguir las comunicaciones que haya en la red. Además de la comunicación con los otros elementos del sistema,
  tiene la opción de conectar, a través de Bluetooth, con un dispositivo Android.
  \item Músico: contiene una mota del tipo ``dispositivo final". Simplemente, recibe (del dispositivo ``director")
  el \textit{tempo} con el que el micromotor vibrador debe activarse (además, al llegar el paquete con la traza de datos, se
  sincroniza con el resto de dispositivos de la red, de forma que todos vibren a la vez).
  \end{itemize}

Como se comentaba cuando se hablaba del dispositivo del director, se puede conectar
con un dispositivo móvil Android. Es necesario que el director indique el tempo que el
sistema debe marcar a los músicos, esta aplicación móvil es la que transfiere al controlador la
velocidad a la que debe compaginar a los intérpretes (para evitar sobrecargar la red, se hace una
coordinación cada cierto tiempo -por si hubiera habido retrasos en la organización inicial- y cada
dispositivo, en función del \textit{tempo} que se le ha enviado, mantiene el pulso). Buscando facilitar al usuario la utilización
de esta tecnología, se ha desarrollado tanto una aplicación para teléfonos móviles.\\

Una vez creada la base del sistema, es posible crear nuevas funcionalidades (como la instalación de un
sensor de vibración que permita a los percusionistas obtener retroalimentación para saber si los intérpretes
están ejecutando la partitura siguiendo el tiempo marcado o la posibilidad de pasar lista -conociendo qué sensores
se encuentran activos en la red-).\\

\cleardoublepage


\thispagestyle{empty}


\begin{center}
{\large\bfseries ArduBand: wearable system to synchronize music bands}\\
\end{center}
\begin{center}
Israel Blancas\\
\end{center}

%\vspace{0.7cm}
\noindent{\textbf{Keywords}: ZigBee, Metronome, Arduino, WSN, XBee, Android}\\

\vspace{0.7cm}
\noindent{\textbf{Abstract}}\\

One of the principles of engineering is creating solutions to the problems of users.
In this dissertation, the reader will be able to find how a wearable system has been developed
helping musicians to go on the same “tempo” through vibration while they are playing music.
This device sends the same pulse to all musicians and keep the “tempo” constant.\\

Some companies have developed similar systems. For example, Peterson created an item called “Body Beat”.
It has a big range of functions and odds but it is too expensive for the majority of musicians. One device
cheaper than it only with synchronization functions (the most important ability of this system) could attract
more buyers. Furthermore, to create a free and open hardware platform could interest other developers to improve
the functionality of this product (helping musicians, music teachers and other music professionals to play music
with a better quality).\\

All music band’s components need to have the same pulse (if each instrumentalist had a different pulse, each
one would read his score in a different speed and it would be a problem). It is possible because it has been
created using a wireless sensor network (they enable communication with a very low energy cost). Then, when all
nodes are synchronized, they know when they have to start the vibrations. But they do not have to be vibrating all
time. They have to vibrate constantly as many times as the director said (for example, if compasses are of 4/4 and
tempo 60 bpm -Beats Per Minute-, each node will vibrate 1 time per second).\\

This network has been deployed using ZigBee implementation of the “Digi International” company. Also, circuit logic
has been put into operation using the hardware platform called Arduino (various versions of this, like Arduino Lilypad,
Arduino Uno or Arduino Leonardo).\\

In the following pages , it is explained in more detail why certain decisions have been taken (some experiments in
time between node's communications, explanations about XBee communication packets...).\\

XBee Zigbee’s devices are organized in mesh network but, changing each node’s configuration, we have now a network with
star topology. There are two kinds of wearables:
  \begin{itemize}
  \item Music director: it is the coordinator of network. It is who start the network and establish the paths of communication packets between all nodes. In addition, it is able to send data to every node and receive data from an Android device via Bluetooth. Only one in each network.
  \item Musician: it is composed of an “end device” node. It receives (from “music director device”) data (which contains “tempo”). Arduino takes this tempo and performs calculations to decide when it has to activate or deactivate a vibration motor (this motor helps musician to keep track of the correct pulse). There is one for each musician (in network, there will be as nodes as musicians).
  \end{itemize}


Director’s device can be connected to an Android device (which could be a smartphone, a tablet or a smartwatch, for example).
This application allows director to indicate music’s
``tempo".\\

Taking this base, it is possible to create new functions such as installing a vibration sensors in drums to measure the
tempo of the band and provide a feedback to the director. Another possibility could be rollcalling at the band (only it
is necessary show what nodes are in the network at the moment).\\


\chapter*{}
\thispagestyle{empty}

\noindent\rule[-1ex]{\textwidth}{2pt}\\[4.5ex]

Yo, \textbf Israel Blancas Álvarez, alumno de la titulación Grado de Ingeniería Informática de la \textbf{Escuela Técnica Superior
de Ingenierías Informática y de Telecomunicación de la Universidad de Granada}, con DNI XXXXXXXXX, autorizo la
ubicación de la siguiente copia de mi Trabajo Fin de Grado en la biblioteca del centro para que pueda ser
consultada por las personas que lo deseen.

\vspace{6cm}

\noindent Fdo: Israel Blancas Álvarez

\vspace{2cm}

\begin{flushright}
Granada a 6 de julio de 2015.
\end{flushright}


\chapter*{}
\thispagestyle{empty}

\noindent\rule[-1ex]{\textwidth}{2pt}\\[4.5ex]

D. \textbf{Samuel Francisco Romero García}, Profesor del Área de Arquitectura y Tecnología de Computadores del Departamento Arquitectura y Tecnología de Computadores de la Universidad de Granada.

\vspace{0.5cm}


\vspace{0.5cm}

\textbf{Informa:}

\vspace{0.5cm}

Que el presente trabajo, titulado \textit{\textbf{ArduBand, vestible inalámbrico para coordinación de músicos en una banda}},
ha sido realizado bajo su supervisión por \textbf{Israel Blancas Álvarez}, y autorizo la defensa de dicho trabajo ante el tribunal
que corresponda.

\vspace{0.5cm}

Y para que conste, expiden y firman el presente informe en Granada a 6 de julio de 2015.

\vspace{1cm}

\textbf{El tutor:}

\vspace{5cm}

\noindent \textbf{Samuel Francisco Romero García}

\chapter*{Agradecimientos}
\thispagestyle{empty}

       \vspace{1cm}

A aquellos maestros y maestras del Colegio Vicente Aleixandre de Granada que siempre creyeron en mí.\\ \\ \\

A los profesores y profesoras del Instituto Mariana Pineda de Granada que me motivaron para
dar cada día lo mejor de mí.\\ \\ \\

A los profesores y profesoras de la ETSIIT que supieron enseñarme a ser más creativo y a proponerme a mí mismo nuevos retos.\\ \\ \\

A mis compañeros de clase por tantas horas de trabajo y momentos, buenos y malos, compartidos.\\ \\ \\

A mis amigos, por que han estado siempre ahí para animarme en los malos momentos.\\ \\ \\

A mi familia, por que, sin su dedicación durante todos estos años, su apoyo en los momentos
de duda y su compresión, no habría llegado a nada.


\setlength{\parskip}{5pt}

\setcounter{tocdepth}{2}
\tableofcontents


\chapter{Introducción y motivación}
\title{Introducción y motivación}
\title{Bandas de música y su problema}
\section{Bandas de música y su problema}

Los directores de orquesta tienen como función la de guiar a los miembros de dicho
grupo en la interpretación de las distintas composiciones (ya sea para realizar correcciones
durante los ensayos, elegir qué obras integrar en el repertorio, aportar un punto de expresividad
en la entonación...).\\

Sin embargo, durante un concierto hay un cometido elemental: otorgar unidad entre los instrumentos
(señalar para que todos los músicos sigan el mismo pulso -y mantener dicha velocidad durante toda la
obra-, por ejemplo).\\

A esto hay que añadirle la dificultad que se presenta cuando bandas de música, profesionales o no,
realizan algún tipo de desfile o pasacalles, donde el director no está visible a todos los músicos y,
por tanto, la tarea descrita anteriormente se hace muy difícil (o imposible) de llevar a cabo.\\

Este trabajo se centrará en tratar de remediar esta problemática.


\title{Soluciones actuales}
\section{Soluciones actuales}

Como principal solución a este problema se generaliza la utilización metrónomos durante los ensayos y,
al actuar en la calle, tratar de conseguir el mismo resultado.\\

Con el despegue de los teléfonos inteligentes, han aparecido múltiples aplicaciones que hacen las veces
de metrónomo (incluso, algunas son capaces de calcular el \textit{tempo} -término que se verá con más detenimiento
después- a partir de las pulsaciones que haga el usuario sobre un botón -esas pulsaciones se deberán hacer
al ritmo que vaya la música-).\\

Por otro lado, los compositores han introducido algún tipo de percusión a sus obras
con la finalidad de favorecer el acompasamiento entre todos los instrumentos (y enriquecer la composición). Si
unimos estos dos hechos, instalando una aplicación de esta naturaleza en un teléfono móvil y éste a su vez en un
soporte para un instrumento de percusión, podríamos mantener la velocidad de interpretación durante la actuación con
un coste relativamente bajo (aunque se mantiene la velocidad en el punto de referencia -que al no ser un computador,
estará sujeto a errores-, no se consigue solucionar totalmente la descoordinación entre los músicos).\\


\title{Conceptos previos necesarios}
\section{Conceptos previos necesarios}
Para poder entender algunos conceptos que se usarán a lo largo de este trabajo y hablar con propiedad,
es necesario tener unos conocimientos musicales mínimos. Se procede a definir algunas nociones:

\begin{itemize}
  \item Pentagrama: es el conjunto formado por cinco líneas horizontales paralelas entre sí y los cuatro espacios que quedan entre ellas.
  Aunque también puede haber líneas adicionales por encima y por debajo del pentagrama, principalmente se usan estas cinco
  líneas y espacios para escribir los símbolos musicales (ya sean notas, silencios...).
    \begin{figure}[htb]
    \centering
    \includegraphics[width=0.8\textwidth]{./imagenes/pentagrama}
    \caption{Representación gráfica de un pentagrama} \label{fig:pentagrama}
    \end{figure}
  \item Pulso: es el latido constante y regular de la música, siendo la unidad temporal básica y, en comparación con
  esta unidad de tiempo, se mide la duración de las notas y silencios.
  \item \textit{Tempo}: es la velocidad del pulso. Para indicar un tempo se utiliza como unidad los “bpm”
    (“\textit{Beats Per Minute}”, es decir, los “Pulsos Por Minuto”). Así, si el tempo de una obra es de 60bpm,
    tendremos que se producirá un pulso por segundo (1bps) o lo que es lo mismo, cada un segundo, tendremos un pulso.
  \item Ritmo: es la combinación de sonidos y silencios de diferente duración.
  \item Compás: facilita la lectura de la música. En un pentagrama, los compases quedan divididos por líneas divisorias.
    Tomaremos que todos los compases son cuaternarios de subdivisión binaria (la mayoría de las obras que interpretan
    el tipo de bandas al que se dirige este producto, se
    encuentran compuestas para este tipo de compases o para compases binarios -encajables en los anteriores- y así podremos simplificar el
    problema para su estudio aunque, como veremos durante la fase de implementación, esto es solo importante para entender mejor
    el diseño -aquí vamos a medir solo \textit{tempo}, el cual es una medida por unidad de tiempo-).
  \end{itemize}

Cabe ahora preguntarse entre qué valores se mueve el tempo, es decir, cual es el rango
del tempo. Existen desde composiciones con menos de 20 pulsaciones por minuto hasta
otras que sobrepasan los 240. Sin embargo, no se puede perder de vista
el público al que se orienta este sistema y, cuyas composiciones, se mueven entre las 70 y 120 pulsaciones
por minuto (dependerá de la composición).\\

Si el lector tiene dudas sobre estos conceptos o desea ampliarlos, puede consultar
la bibliografía \cite{teorMusica} \cite{lenguajeMusical}.\\


\title{Producto a desarrollar}
\section{Producto a desarrollar}

Teniendo en cuenta todo lo dicho en las anteriores páginas, es el momento de manifestar
qué dispositivo es el que se desea desarrollar en este trabajo: un sistema que marque el
pulso (que no el ritmo, al poder ser éste irregular mientras que el pulso es constante)
en función del \textit{tempo} que indique el director de la agrupación. Además, el sistema deberá
ser discreto ya que se busca que las bandas lo utilicen principalmente en la calle.\\

Esta necesidad por parte de las bandas de música ha sido detectada por algunos fabricantes,
como Peterson, que puso a la venta un producto llamado “Body Beat”. Dispone de un amplio
abanico de funciones, pero su tamaño y su elevado coste hacen inviable la implantación del
sistema en una banda de música (hay que tener en cuenta que el número de componentes en una
banda es variable pero ronda entre los 50 y 100 músicos -algunas de ellas sobrepasan este número,
como la ``Banda de Cornetas y Tambores Nuestra Señora de la Victoria"\ conocida como ``Las Cigarreras"\
de Sevilla \cite{cigarreras} que, en las fechas en las que se escribe este trabajo, ronda los 140 componentes-,
por lo que no sería un coste asumible. Por otro lado, las dimensiones, de unos 10.8 cm x 7.6 cm x 2.54 cm puede
que sean demasiado grandes).
Como último punto a tener en cuenta, muchos son los usuarios que, a través de diversas páginas en Internet,
se lamentan de la corta duración de la batería (teniendo en cuenta que hay actuaciones que pueden llegar a
durar entre 8 y 10 horas, esto es un problema importante).\\


\begin{figure}[htb]
\centering
\captionsetup{justification=centering}
\includegraphics[width=0.6\textwidth]{./imagenes/bodybeat}
\caption{``Body Beat" de la marca Peterson.\\
\scriptsize{Imagen extraída de http://www.petersontuners.com} \label{fig:bodybeat}}
\end{figure}

Un dispositivo más barato con un número menor de funciones pero que permita la sincronización de todos
los dispositivos y mantener el tempo durante toda la interpretación, atraería más usuarios. Si además
se procura que la construcción se haga utilizando software y hardware libre, podría crearse una comunidad
de desarrolladores en torno al producto, consiguiendo mejorar la calidad del dispositivo y aumentar la funcionalidad
de este.\\

Las principales fuentes consultadas han sido:
\begin{itemize}
  \item ``Building Wireless Sensor Networks", Robert Faludi. Editorial: O’Reilly. \cite{faludi}
  \item ``Blog de Robert Faludi", Robert Faludi. \cite{faludiBlog}
  \item ``Ingeniería del software: un enfoque práctico", Roger S. Pressman \cite{pressman}
  \item ``Web oficial de Arduino" \cite{arduinoWeb}
  \item ``Google Developers" \cite{googledevelopers}
  \item ``Sistemas de Tiempo Real y Lenguajes de Programación", Alan Burns y Andy Wellings. Editorial: ADDISON-WESLEY \cite{sistemastiemporealyprogramacion}
\end{itemize}

%
\chapter{Objetivos}
\title{Objetivos}

\title{Objetivos principales}
\section{
Objetivos principales
}
Se desea desarrollar un \textbf{sistema que permita sincronizar a los músicos de una banda} entre sí,
utilizando \textbf{tecnologías libres}. Debido al número de horas que puede llegar a durar una
actuación, se quiere que tenga un \textbf{bajo consumo energético} (y conseguir que la vida de la batería se alargue).
También se quiere que sea
\textbf{vestible}, de forma que sea cómodo y discreto para el usuario. Por supuesto, para que
sea competitivo, debe ser \textbf{más barato que otras soluciones} disponibles en el mercado y
\textbf{escalable en el número de dispositivos}. Se pretente, además, que sea \textbf{posible añadir nuevas
funciones a posteriori}.\\

Se desea utilizar tecnologías libres para que cualquier desarrollador pueda aportar su conocimiento al proyecto, colaborando
con aquellas bandas de música que utilicen el producto. Además, como ya se dijo en la introducción,
una comunidad alrededor de este producto podría mejorar el rendimiento del mismo y añadir nuevas funcionalidades.\\

El hecho de que el sistema sea vestible viene por dos razones: comodidad (los músicos llevarían mucho tiempo encima el aparato y,
unido al instrumento y partituras, un dispositivo que sea demasiado grande, será un lastre) y decoro (no queda bien de cara al público que
un músico vestido de gala lleve un aparato que sea demasiado llamativo).\\

La sincronización que se pretende llevar a cabo es la solución al problema planteado durante la introducción: es complicado
mantener acompasados a todos los músicos de una banda mientras se está realizando una actuación en la calle y el director no puede
realizar adecuadamente sus funciones.\\

Los objetivos no se encuentran muy interconectados. El bajo consumo energético y que sea vestible se encuentran
relacionados (si tiene un consumo energético muy alto, se pasará más tiempo dependiendo de la conexión a una fuente
de alimentación no portable del que debiera). Por otro lado, al querer construir un sistema que permita comunicación,
será necesario que sea escalable (cuanta más escalabilidad permita, mayor será el número de personas que podamos conectar).

Todos los objetivos se tratan durante el capítulo 7 (correspondiente a la implementación del sistema) excepto
la creación de un dispositivo más barato que otros productos del mercado (se hablará de esto en los capítulos 4 -cuando se
estimen costes- y en el capítulo 8 -donde se comparará con otro sistema existente en el mercado-) y la posibilidad de añadir
nuevas funciones (que se tratará en el capítulo 8).


\title{Consideraciones técnicas}
\section{Consideraciones técnicas}

  \begin{itemize}
    \item Una posible solución al problema planteado puede ser el despliegue de una red inalámbrica de sensores,
    más teniendo en cuenta la actual dirección de la industria respecto a este tipo de tecnología (su aplicación,
    por ejemplo en el “Internet de las Cosas”, que se encuentra en la cima de la curva de Gartner \ref{fig:curvaG})
    \item La creación de dispositivos \textit{wearables} es algo que también ha suscitado
    mucha atención por parte de la industria (como también se puede ver en la curva de Gartner)
  \end{itemize}

  \begin{figure}[htb]
  \centering
  \captionsetup{justification=centering}
  \includegraphics[width=1\textwidth]{./imagenes/gartner}
  \caption{Curva de Gartner (agosto de 2014)\\
    \scriptsize{Imagen extraída de http://www.gartner.com/}
  \cite{gartnercurve}} \label{fig:curvaG}
  \end{figure}

La Gartner es una empresa consultora especializada en nuevas tecnologías. La curva de Gartner es
un estudio realizado por esta empresa en el que se analizan distintas tecnologias
y su estado en el momento de la investigación (fase de lanzamiento, pico
de altas expectativas, valle de la desilusión, rampa de la consolidación o meseta
de la productividad).\\

\title{Aspectos formativos previos}
\section{Aspectos formativos previos}

Aunque todo el conocimiento obtenido durante el grado ha sido importante para el desarrollo del proyecto, hay que destacar:
  \begin{itemize}
    \item Fundamentos de Ingeniería del Software: para establecer los requisitos,
    la planificación, costes de desarrollo...
    \item Programación de Dispositivos Móviles: para realizar las aplicaciones ``Android"
    \item Tecnologías Emergentes: en esta asignatura se ha impartido materia sobre el uso
    de placas controladoras, dispositivos \textit{wearables} y redes inalámbricas de sensores
    \item Informática Industrial: algunos conocimientos sobre redes inalámbricas de sensores
    \item Fundamentos de Redes: para comprender las distintas capas de los protocolos analizados y
    la estructura de los paquetes que se envían entre sí los dispositivos
  \end{itemize}

%
\chapter{Especificación de Requisitos}
\label{cap:EspecificaciondeRequisitos}
\title{Especificación de Requisitos}

Para lograr que el desarrollo de un sistema hardware o software sea un éxito,
es necesario que los desarrolladores comprendan totalmente las especificaciones que
requiere el proyecto. En esta sección se va a proceder a enumerar los requisitos del
sistema en el que se está trabajando.\\

\title{Modelado del sistema}
\section{
Modelado del sistema
}

El dispositivo que se desea desarrollar tiene que hacer la función básica
del director de la banda de música durante una actuación, es decir,
marcar el mismo pulso a los músicos.\\

Cuando la agrupación se encuentra realizando un concierto, el
esquema de comunicación que se sigue es el mostrado en la figura \ref{fig:modeladoconceptual}
(donde el nodo rojo es el director y lo morados los músicos). El único flujo de información
que hay es que que marcan las flechas (el pulso, que sigue un tempo constante). Hay que
aclarar que en este caso, para simplificar, sólo se han dibujado 5 músicos,
sin embargo, como se vio en la introducción, las bandas suelen contar entre sus filas con
varias decenas de integrantes.\\


\begin{figure}[htb]
\centering
\includegraphics[width=0.6\textwidth]{./imagenes/modeladoconceptual}
\caption{Modelado del sistema} \label{fig:modeladoconceptual}
\end{figure}

El flujo de información es el mostrado en la figura \ref{fig:mensajesconceptual}
  \begin{figure}[htb]
  \centering
  \includegraphics[width=0.6\textwidth]{./imagenes/mensajesconceptual}
  \caption{Traspaso de información} \label{fig:mensajesconceptual}
  \end{figure}


Es necesario guardar el ``tempo'' (que será establecido por el director) y algún tipo
de idenficación de los músicos, para que el director sepa con quién se tiene que
comunicar (aunque esto se verá más adelante).\\


\title{Actores}
\section{Actores}
\label{sec:actoresRequisitos}

Del esquema de la figura \ref{fig:mensajesconceptual} podemos sacar una conclusión clara:
será necesaria la distinción entre dos tipos de usuarios en nuestro sistema.
  \begin{itemize}
    \item Director: es el que envía el pulso al resto de actores. Solo uno por sistema
    \item Músico: recibe el pulso del director. Habrá multiples
  \end{itemize}


\title{Requisitos funcionales}
\section{Requisitos funcionales}

En este apartado se van a describir los requisitos funcionales del sistema.
Estos requisitos son aquellas funciones o capacidades que el sitema debe
tener para satisfacer las necesidades de los usuarios.\\

\begin{itemize}
    \item[\textbf{RF.1}] El sistema permitirá al director insertar un tempo
    \item[\textbf{RF.2}] Se capacitará al director para hacer llegar el pulso adecuado a los intérpretes
    \item[\textbf{RF.3}] El pulso se comunicará a los músicos a través de algún actuador (una señal visual, por ejemplo)
\end{itemize}


\title{Requisitos no funcionales}
\section{Requisitos no funcionales}

En estea sección se van a describir los distintos requisitos no funcionales del sistema.
Los requisitos no funcionales son aquellos que nos informan de restricciones que debe cumplir
nuestro sistema.\\


\begin{itemize}
    \item[\textbf{RNF.1}] El sistema debe ser wireless: la comunicación entre los
      dispositivos debe hacerse sin cables
    \item[\textbf{RNF.2}] Debe de ser barato: para que el usuario desee utilizar
      el sistema, este debe tener un coste asumible
    \item[\textbf{RNF.3}] La escalabilidad no debe ser un problema: se deben poder añadir
      muchos dispositivos al sistema y que la funcionalidad no se resienta.
    \item[\textbf{RNF.4}] Debe permitir que se puedan añadir funciones: para futuras mejoras.
    \item[\textbf{RNF.5}] El dispositivo que se desarrolle tiene que ser discreto y cómodo:
      ningún usuario va a querer utilizarlo si se siente mal llevándolo o si es demasiado llamativo.
    \item[\textbf{RNF.6}] Bajo consumo energético: se busca aprovechar la batería al máximo
      (si el dispositivo funciona durante poco tiempo, no habrá usuarios que deseen utilizarlo).
    \item[\textbf{RNF.7}] Sincronización entre los músicos: ya que lo que se busca es dar unidad
      a la interpretación de las distintas obras, el sistema debe cumplir unos requisitos de tiempo
      importantes (a nivel de milisegundos para que, en caso de haber algo de asincronía, el usuario no lo note).
    \item[\textbf{RNF.8}] El pulso debe mantenerse constante: el pulso, por definición
      no cambia a lo largo de la obra y debe conseguirse que el usuario lo perciba de forma
      constante
\end{itemize}


\title{Requisitos de información}
\section{Requisitos de información}

Estos requisitos indican qué información guarda nuestro sistema.\\

  \begin{itemize}
    \item[\textbf{RI.1}] Tempo: se guardará el tempo indicado por el director
    para poder transmitir el pulso adecuado a los distintos músicos o, en su defecto,
    enviar el tempo una vez que estén sincronizados
    \item[\textbf{RI.2}] Músicos: el director deberá conocer los músicos existentes en la banda
    para saber a quién tiene que comunicarle el tempo. Podría darse el caso de dos
    agrupaciones musicales que se encontrasen muy próximas y utilizase en sistema:
    hay que evitar que el pulso de una de las bandas interfiera el de la otra
  \end{itemize}

%
\chapter{Planificación}
\title{Planificación}

\title{Metodología de desarrollo}
\section{Metodología de desarrollo}

\paragraph{
La metodología que se ha tratado de seguir para el desarrollo de este proyecto ha sido
un modelo en espiral ya que lo primero que se quiere solucionar es una funcionalidad
básica (mantener el mismo pulso entre todos los músicos). Sobre esto,
se desea que la comunidad (o los mismos desarrolladores), introduzcan nuevas
funcionalidades.
}

\paragraph{
Incluso dentro del desarrollo de la funcionalidad más básica (que es
la que se trata de implementar en este trabajo), se sigue el una metodología en espiral:
}

\begin{enumerate}
  \item Extracción de los requisitos
  \item Planificación
  \item Ingeniería
  \item Construcción
  \item Evaluación
\end{enumerate}

\title{Fases}
\section{Fases}
\paragraph{
En la anterior sección se ha comentado la metodología a utilizar y, de forma general,
las fases a desarrollar.
}

\begin{enumerate}
  \item Extracción de los requisitos: esto se ha hecho a lo largo del capítulo \ref{cap:EspecificaciondeRequisitos}
  \item Planificación: es lo que se está tratando en este mismo capítulo
  \item Ingeniería: se hablará de esto en los capítulos  \ref{cap:Analisis} y  \ref{cap:Diseno}
  \item Construcción
    \begin{enumerate}
      \item Comunicación entre los actores del sistema
      \item Cálculo de los valores necesarios para mantener el tempo en un actor
      \item Envío del tempo desde el director a los músicos
      \item Posibilidad de cambiar el tempo (por parte del director)
      \item Aplicaciones propias para cambiar el tempo
    \end{enumerate}
  \item Evaluación
\end{enumerate}



\title{Recursos humanos}
\section{Recursos humanos}

\paragraph{
Gracias a que es un proyecto de software/hardware libre, todo aquel que
desee particiar en el proyecto puede hacerlo realizando aportaciones en los
\href{https://github.com/iblancasa/ArduBand}{repositorios del proyecto}.
}


\title{Recursos reutilizables}
\section{Recursos técnicos reutilizables}

\paragraph{
Estos recursos hacen referencia a anteriores desarrollos hardware o software creados
por otros desarrolladores o empresas.
}

\paragraph{
En el capítulo de implementación se explicará con más detalle por qué se
eligen los elementos que se van a describir en los dos siguientes apartados
}


\subsection{Recursos software reutilizables}
\title{Recursos software reutilizable}


\paragraph{
En la introducción se hacía hincapié en la necesidad de crear una aplicación
Android para ayudar al director a establecer el tempo deseado. Por tanto,
serán recursos a tener en cuenta durante el desarrollo todas las herramientas
que implementa el propio SDK de Android y aquellas bibliotecas con licencia
libre que se encuentran en Internet (ya sean para añadir funcionalidad o elementos
gráficos).
}

\paragraph{
También se va a desarrollar un dispositivo físico que necesitará un software para estar
controlado. Todas las bibliotecas del controlador con licencia libre, también estarán
disponibles para su uso (en la sección de implementacion, veremos que se utilizará
Arduino -el cual tiene muchas bibliotecas disponibles- y XBee -existiendo para este
sistema de comunicación Wireless algunas bibliotecas desarrolladas por la comunidad-).
}


\subsection{Recursos hardware reutilizables}
\title{Recursos hardware reutilizable}

\paragraph{
Como se acaba de comentar, se utilizará la plataforma Arduino. Además de poder comprar
placas ya fabricadas, existe la posibilidad de crear una propia (es hardware libre y
los planos y software se encuentran disponibles en GitHub \cite{arduinoRepo}).
}

\paragraph{
Se utilizará también XBee, de la compañía Digi \cite{xbeedatasheet}.
}


\paragraph{
Por otro lado, se puede considerar el dispositivo móvil Android como un elemento a
tener en cuenta en esta sección (ya que es el elemento que hará de interfaz con el usuario,
eliminando la necesidad de crear un dispositivo físico para que el director interactúe con
el sistema).
}

%
\chapter{Análisis}
\title{Análisis}
\label{cap:Analisis}

\paragraph{
aasd
}

%
\chapter{Diseño}
\title{Diseño}
\label{cap:Diseno}

\section{Arquitectura}
Como se ha ido explicando en los capítulos anteriores, el dispositivo que se desea desarrollar
tiene que hacer la función básica del director de la banda de música durante una actuación,
es decir, marcar el mismo pulso a los músicos.\\

Si recordamos el esquema planteado en la figura \ref{fig:modeladoconceptual} y todo lo
que se ha ido desarrollando a través de los capítulos anteriores, podemos decir
que el sistema sigue una arquitectura de repositorio -o pizarra- (siendo el repositorio activo):

\begin{itemize}
  \item El director es el centro de todo. Se encarga de realizar la comunicación con los músicos
  \item Los músicos son subsistemas independientes (aunque tengan que estar sincronizados)
  \item Alto acoplamiento entre los músicos y el director
\end{itemize}

A parte de la relación existente entre músicos y director, no debemos olvidar que
deseamos comunicarnos desde un dispositivo (Android según las especificaciones pero que,
como se verá en la implementación, valdrá cualquier SO). Es por ello que el diagrama
de despliegue queda como en la figura \ref{fig:diagramadespliegue}.\\

\section{Diagrama conceptual}
\title{Diagrama conceptual}
Este diagrama (que podemos ver en la figura \ref{fig:modeloconceptual})
nos da una idea de cómo se encuentran interrelacionados los diferentes
agentes del sistema. En nuestro caso, podemos ver cómo un ``Director"\ está asociado
a muchos ``Músicos"\ (o ninguno) y un ``Músico"\ solo tiene asignado un director.\\


\begin{figure}[!htb]
\centering
\includegraphics[width=1\textwidth]{./imagenes/modeloconceptual}
\caption{Modelo conceptual} \label{fig:modeloconceptual}
\end{figure}

\begin{figure}[htb]
\centering
\includegraphics[width=1\textwidth]{./imagenes/diagramadespliegue}
\caption{Diagrama de despliegue} \label{fig:diagramadespliegue}
\end{figure}

\section{Diagrama de secuencia del sistema}
Para ver desde otra perspectiva el funcionamiento del sistema, se ha creado el diagrama
de secuencia de la figura \ref{fig:diagramasecuencia}\\


\begin{figure}[!htb]
\centering
\includegraphics[width=1\textwidth]{./imagenes/diagramasecuencia}
\caption{Diagrama de secuencia} \label{fig:diagramasecuencia}
\end{figure}


\section{Diagramas de comunicación}
\title{Diagramas de comunicación}

Estos diagramas permiten conocer la relación que hay entre los distintos
conceptos del sistema y qué información pasa de unos a otros.\\

Los casos de uso expuestos en \ref{subsec:casosdeuso} se han diseñado según
se muestra en las figuras \ref{fig:comunicacion1},  \ref{fig:comunicacion2} y
\ref{fig:comunicacion3}.\\

\begin{figure}[htb]
\centering
\includegraphics[width=1\textwidth]{./imagenes/comunicacion1}
\caption{Diagrama de comunicación 1} \label{fig:comunicacion1}
\end{figure}

\begin{figure}[htb]
\centering
\includegraphics[width=1\textwidth]{./imagenes/comunicacion2}
\caption{Diagrama de comunicación 2} \label{fig:comunicacion2}
\end{figure}

\begin{figure}[htb]
\centering
\includegraphics[width=1\textwidth]{./imagenes/comunicacion3}
\caption{Diagrama de comunicación 3} \label{fig:comunicacion3}
\end{figure}

\section{Interfaces}
\title{Interfaces}

Como medio de interacción entre el director y el dispositivo, se va a utilizar,
como se ha venido hablando a lo largo de la memoria, un dispositivo Android y/o
Android Wear. En la implementación se verá exactamente qué sistema utilizar
para realizar la comunicación.\\

En cuanto a la interfaz entre el dispositivo y el músico, ya se mencionaba en
los primeros capítulos de esta memoria la intención de utilizar algún actuador
para avisar al músico.\\


\subsection{Interfaz músico-sistema}
\title{Interfaz músico-sistema}

La interfaz músico-sistema será por \textit{hardware}. Como se ha ido adelantando
en otras secciones del proyecto, se usará un actuador para comunicar al músico
la velocidad a la que se está interpretando la composición.\\
Ya que no se quiere que sea algo muy llamativo, se optará
por un micromotor vibrador (en la implementación se elegirá cuál).\\

\begin{figure}[!htb]
\centering
\includegraphics[width=0.7\textwidth]{./imagenes/motorarduband}
\caption{Interfaz músico-sistema} \label{fig:motorarduband}
\end{figure}

\subsection{Interfaces gráficas}
\title{Interfaces gráficas}

Corresponde con la interfaz director-sistema.\\

\begin{figure}[htb]
\centering
\includegraphics[width=1\textwidth]{./imagenes/movilarduband}
\caption{Esquema de comunicación entre el dispositivo móvil y ArduBand} \label{fig:movilarduband}
\end{figure}

El esquema que se desea seguir es el de la figura \ref{fig:movilarduband}.

Con objeto de tener una guía para de cara a la implementación, se ha planteado un
prototipo del diseño de la interfaz gráfica que se puede ver en la figura
\ref{fig:prototipointerfaz} (realizado con el software \href{http://pencil.evolus.vn/}{Pencil}).\\

\begin{figure}[htb]
\centering
\includegraphics[width=1\textwidth]{./imagenes/prototipointerfaz}
\caption{Prototipo de interfaz} \label{fig:prototipointerfaz}
\end{figure}

En la figura \ref{fig:prototipointerfaz} podemos observar tres pantallas (dos de de ellas
pertenecientes a teléfono móvil y una tercera a un reloj Android Wear).

\begin{description}
  \item[Android] \hfill \\
    Se ha buscado que sea una interfaz lo más simple posible para que el usuario no dude
    en su uso. En el centro se ha insertado un campo para que se añada el tempo y un botón de aplicar.
    Se ha añadido una sección ``Sobre"\ en el que se han colocado dos enlaces (que
    llevan a la web del autor y al repositorio del proyecto) y el logotipo de la aplicación
  \item[Android Wear] \hfill \\
    Más sencilla que la anterior. En grande se muestra el tempo. Hay dos botones en la parte inferior que permiten
    aumentar o disminuir la cifra de tempo. Cuando se desee aplicar, se pulsa encima del tempo y el sistema
    comienza a funciona
\end{description}

Cuando se realice la implementación, se tratará de seguir lo más fielmente posible las directrices de diseño que
Google da en su guía para desarrolladores (\textit{Material Design}) \cite{googlematerial}.\\

\section{Carcasa}
\title{Carcasa}
Aunque no sea la versión final, se ha realizado un prototipo de lo que podría ser la carcasa para el dispositivo,
y que se adjunta en la figura \ref{fig:prototipocase}. \\

\begin{figure}[htb]
\centering
\includegraphics[width=0.7\textwidth]{./imagenes/prototipo}
\caption{Prototipo de carcasa} \label{fig:prototipocase}
\end{figure}

%
\chapter{Implementación}
\title{Implementación}
\label{cap:Implementacion}

\paragraph{
aasd
}

%
%\input{capitulos/08_Pruebas}
%
%\input{capitulos/09_Conclusiones}
%
%\input{capitulos/10_Conclusiones y trabajos futuros}
%%\chapter{Conclusiones y Trabajos Futuros}
%
%
\nocite{*}
\bibliography{bibliografia}\addcontentsline{toc}{chapter}{Bibliografía}
\bibliographystyle{unsrt}


\renewcommand*\listfigurename{Lista de figuras}
\listoffigures

\renewcommand{\listtablename}{Índice de tablas}
\listoftables


\renewcommand{\listalgorithmname}{Lista de algoritmos}
\listofalgorithms


\chapter*{}
\thispagestyle{empty}

\end{document}
