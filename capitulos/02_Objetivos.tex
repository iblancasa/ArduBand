\chapter{Objetivos}
\title{Objetivos}

\title{Objetivos principales}
\section{
Objetivos principales
}
  \begin{itemize}
    \item[\textbf{OBJ.1}] Que el sistema sea wireless: se quiere realizar un dispositivo que permita la comunicación con el resto de dispositivos del sistema sin necesidad de una conexión física entre ellos
    \item[\textbf{OBJ.2}] Conseguir un precio menor que otras soluciones del mercado: utilizando una tecnología distinta a la que han usado otros productos, tratar de obtener un sistema con un menor costo
    \item[\textbf{OBJ.3}] Escalable en el número de dispositivos: se quiere crear un sistema que disponga de una gran escalabilidad
    \item[\textbf{OBJ.4}] Ampliable en funciones: posibilidad de desarrollar nuevas funcionalidades partiendo de la funcionalidad más básica del sistema
    \item[\textbf{OBJ.5}] Vestible: debe ser un sistema discreto y cómodo para el portador
    \item[\textbf{OBJ.6}] Bajo consumo energético: se quiere desarrollar un sistema que no consuma demasiada energía para sacar el máximo partido a la batería que se inserte
    \item[\textbf{OBJ.7}] Tecnología lo más libre posible: se desea utilizar herramientas libres para tratar de atraer a desarrolladores para que colaboren en el proyecto. Además por la propia naturaleza del software/hardware libre, la comunidad aportará parches y soluciones a los problemas que puedan presentarse, mejorando la calidad del desarrollo
    \item[\textbf{OBJ.8}] Suficiente para cubrir las necesidades del mercado: aunque anteriormente se mencionaba que se desea que el sistema sea ampliable en funciones, es también necesario que la versión inicial tenga unas funciones mínimas que permitan cubrir las necesidades básicas del mercado
  \end{itemize}

\title{Objetivos secundarios}
\section{
Objetivos secundarios
}
  \begin{itemize}
    \item[\textbf{obj.1}]Utilizar diversas versiones de la plataforma hardware Arduino: se tiene como objetivo utilizar distintas versiones de Arduino para poder obtener el producto con distintos formatos (Arduino Lilypad, Arduino Uno...)
    \item[\textbf{obj.2}]Desarrollar una red inalámbrica de sensores: teniendo en cuenta la actual dirección de la industria respecto a este tipo de tecnología (su aplicación, por ejemplo. en el “Internet de las Cosas” \cite{hypeIoT} \cite{gatnercurve}) , es interesante trabajar con esta tecnología
    \item[\textbf{obj.3}]Crear un dispositivo wearable: actualmente es uno de los sectores en los que más están trabajando las compañías. Si miramos la curva de Gatner \cite{gatnercurve}, en el año 2014 estas tecnologías se situaban en la cima del ciclo
  \end{itemize}

\title{Aspectos formativos previos}
\section{
Aspectos formativos previos
}

Aunque todo el conocimiento obtenido durante el grado ha sido importante para el desarrollo del proyecto, para el desarrollo del proyecto, hay que destacar:

  \begin{itemize}
    \item Conocimientos básicos de Ingeniería del Software: para establecer los requisitos, la planificación, costes de desarrollo...
    \item Nociones en programación: en especial las adquiridas en cuanto a la programación de dispositivos móviles
    \item Tecnologías Emergentes: en esta asignatura se ha impartido materia sobre el uso de placas controladoras, dispositivos wearables y redes inalámbricas de sensores (también vistas en la asignatura Informática Industrial)
  \end{itemize}
