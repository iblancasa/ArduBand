\chapter{Planificación}
\title{Planificación}

\title{Metodología de desarrollo}
\section{Metodología de desarrollo}

\paragraph{
La metodología que se ha tratado de seguir para el desarrollo de este proyecto ha sido
un modelo en espiral ya que lo primero que se quiere solucionar es una funcionalidad
básica (mantener el mismo pulso entre todos los músicos). Sobre esto,
se desea que la comunidad (o los mismos desarrolladores), introduzcan nuevas
funcionalidades.
}

\paragraph{
Incluso dentro del desarrollo de la funcionalidad más básica (que es
la que se trata de implementar en este trabajo), se sigue el una metodología en espiral:
}

\begin{enumerate}
  \item Extracción de los requisitos
  \item Planificación
  \item Ingeniería
  \item Construcción
  \item Evaluación
\end{enumerate}

\title{Fases}
\section{Fases}
\paragraph{
En la anterior sección se ha comentado la metodología a utilizar y, de forma general,
las fases a desarrollar.
}

\begin{enumerate}
  \item Extracción de los requisitos: esto se ha hecho a lo largo del capítulo \ref{cap:EspecificaciondeRequisitos}
  \item Planificación: es lo que se está tratando en este mismo capítulo
  \item Ingeniería: se hablará de esto en los capítulos  \ref{cap:Analisis} y  \ref{cap:EspecificaciondeRequisitos}
  \item Construcción
    \begin{enumerate}
      \item Comunicación entre los actores del sistema
      \item Cálculo de los valores necesarios para mantener el tempo en un actor
      \item Envío del tempo desde el director a los músicos
      \item Posibilidad de cambiar el tempo (por parte del director)
      \item Aplicaciones propias para cambiar el tempo
    \end{enumerate}
  \item Evaluación
\end{enumerate}



\title{Recursos humanos}
\section{Recursos humanos}

\paragraph{
Gracias a que es un proyecto de software/hardware libre, todo aquel que
desee particiar en el proyecto puede hacerlo realizando aportaciones en los
\href{https://github.com/iblancasa/ArduBand}{repositorios del proyecto}.
}


\title{Recursos reutilizables}
\section{Recursos técnicos reutilizables}

\paragraph{
Estos recursos hacen referencia a anteriores desarrollos hardware o software creados
por otros desarrolladores o empresas.
}

\paragraph{
En el capítulo de implementación se explicará con más detalle por qué se
eligen los elementos que se van a describir en los dos siguientes apartados
}


\subsection{Recursos software reutilizables}
\title{Recursos software reutilizable}


\paragraph{
En la introducción se hacía hincapié en la necesidad de crear una aplicación
Android para ayudar al director a establecer el tempo deseado. Por tanto,
serán recursos a tener en cuenta durante el desarrollo todas las herramientas
que implementa el propio SDK de Android y aquellas bibliotecas con licencia
libre que se encuentran en Internet (ya sean para añadir funcionalidad o elementos
gráficos).
}

\paragraph{
También se va a desarrollar un dispositivo físico que necesitará un software para estar
controlado. Todas las bibliotecas del controlador con licencia libre, también estarán
disponibles para su uso (en la sección de implementacion, veremos que se utilizará
Arduino -el cual tiene muchas bibliotecas disponibles- y XBee -existiendo para este
sistema de comunicación Wireless algunas bibliotecas desarrolladas por la comunidad-).
}


\subsection{Recursos hardware reutilizables}
\title{Recursos hardware reutilizable}

\paragraph{
Como se acaba de comentar, se utilizará la plataforma Arduino. Además de poder comprar
placas ya fabricadas, existe la posibilidad de crear una propia (es hardware libre y
los planos y software se encuentran disponibles en GitHub \cite{arduinoRepo}).
}

\paragraph{
Se utilizará también XBee, de la compañía Digi \cite{xbeedatasheet}.
}


\paragraph{
Por otro lado, se puede considerar el dispositivo móvil Android como un elemento a
tener en cuenta en esta sección (ya que es el elemento que hará de interfaz con el usuario,
eliminando la necesidad de crear un dispositivo físico para que el director interactúe con
el sistema).
}
