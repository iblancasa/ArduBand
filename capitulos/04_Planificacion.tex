\chapter{Planificación}
\title{Planificación}

\title{Metodología de desarrollo}
\section{Metodología de desarrollo}

\begin{figure}[htb]
\centering
\includegraphics[width=1\textwidth]{./imagenes/modeloespiral}
\caption{Metodología de desarrollo en espiral} \label{fig:modeloespiral}
\end{figure}

La metodología que se ha tratado de seguir para el desarrollo de este proyecto ha sido
un modelo en espiral ya que lo primero que se quiere solucionar es una funcionalidad
básica (mantener el mismo pulso entre todos los músicos). Sobre esto,
se desea que la comunidad (o los mismos desarrolladores), introduzcan nuevas
funcionalidades (la espiral seguiría girando, empezando otra vez por la primera etapa).\\

Incluso dentro del desarrollo de la funcionalidad más básica (que es
la que se trata de implementar en este trabajo), se sigue el una metodología en espiral:

\begin{enumerate}
  \item Extracción de los requisitos
  \item Planificación
  \item Ingeniería
  \item Construcción
\end{enumerate}


\title{Fases}
\section{Fases}
En la anterior sección se ha comentado la metodología a utilizar y, de forma general,
las fases a desarrollar. En esta sección se va a entrar en más detalle.

\begin{description}
  \item [Extracción de los requisitos]\hfill \\
  \begin{itemize}
    \item Descripción: obtención de los requisitos funcionales, no funcionales y de información del proyecto
    \item Apartado: Capítulo \ref{cap:EspecificaciondeRequisitos}
  \end{itemize}
  \item [Planificación]\hfill \\
    \begin{itemize}
      \item Descripción: estimación de costo, temporal, recursos humanos...
      \item Apartado: Este capítulo
    \end{itemize}
  \item [Ingeniería]\hfill \\
  \begin{itemize}
    \item Descripción: análisis de los requisitos y diseño del sistema a desarrollar
    \item Apartado: Capítulos \ref{cap:Analisis} y  \ref{cap:Diseno}
  \end{itemize}
  \item [Construcción]\hfill \\
  \item Descripción: implementación del proyecto
  \item Apartado: Capítulo \ref{cap:Implementacion}
\end{description}

\title{Estimación temporal}
\section{Estimación temporal}
Teniendo en cuenta las partes detalladas en las sección anterior, es momento de
hacer una estimación temporal del proyecto.\\


\begin{itemize}
  \item Especificaciones del proyecto
  \begin{itemize}
    \item{Conocer adecuadamente las necesidades del usuario}
    \item{Establecer los objetivos del proyecto}
    \item{Extraer requisitos funcionales}
    \item{Extraer requisitos no funcionales}
    \item{Extraer requisitos de información}
    \item{Tiempo estimado: \textbf{15 horas}}
  \end{itemize}
\end{itemize}

\begin{itemize}
  \item Planificación
  \begin{itemize}
    \item{Calcular un presupuesto}
    \item{Establecer una temporización}
    \item{Establecer las fases}
    \item{Encontrar recursos que puedan ser reutilizables en el proyecto}
    \item{Definir los recursos humanos disponibles}
    \item{Tiempo estimado: \textbf{20 horas}}
  \end{itemize}
\end{itemize}

\begin{itemize}
  \item Análisis y diseño
  \begin{itemize}
    \item{Analizar los requisitos del proyecto}
    \item{Creación de diagramas}
    \item{Diseño de arquitectura}
    \item{Tiempo estimado: \textbf{40 horas}}
  \end{itemize}
\end{itemize}


\begin{itemize}
  \item Construcción
  \begin{itemize}
    \item{Establecer herramientas/plataformas/lenguaje/software a utilizar}
      \begin{itemize}
        \item{Decidir qué plataformas utilizar}
        \item{Investigar las herramientas de desarrollo de las plataformas}
        \item{Realizar pruebas para conocer si las plataformas son válidas para el proyecto}
      \end{itemize}
    \item{Comunicación entre los actores del sistema}
    \item{Cálculo de los valores necesarios para mantener el tempo en un actor}
    \item{Envío del tempo desde el director a los músicos}
    \item{Posibilidad de cambiar el tempo (por parte del director)}
    \item{Aplicaciones propias para cambiar el tempo}
    \item{Tiempo estimado: \textbf{100 horas}}
  \end{itemize}
\end{itemize}

\begin{itemize}
  \item Documentación
  \begin{itemize}
    \item{Documentar código}
    \item{Documentación del proyecto}
    \item{Creación de esquemas eléctricos para ayudar a futuros desarrolladores
    a comprender el conexionado}
    \item{Tiempo estimado: \textbf{40 horas}}
  \end{itemize}
\end{itemize}

\begin{table}[h]
\centering
\label{table:estimadobruto}
\begin{tabular}{ll}
\hline
\rowcolor[HTML]{9698ED}
{\bf Tarea}                   & {\bf Tiempo estimado (horas)} \\ \hline
Especificaciones del proyecto & 15                            \\
Planificación                 & 20                            \\
Análisis y diseño             & 40                            \\
Construcción                  & 100                           \\
Documentación                 & 40                            \\
\rowcolor[HTML]{CBCEFB}
{\bf Total}                   & 215                           \\ \hline
\end{tabular}
\caption{Estimación temporal}
\end{table}

Se ha hecho una estimación de la cantidad de días que ocuparían todas
estas tareas y, para mostrarlo de una forma más gráfica,
se ha creado un diagrama de Gantt, que se puede ver en la figura \ref{fig:gantt} y que sigue la siguiente leyenda:

\begin{itemize}
  \item  \textcolor{blueS}{$\blacksquare$} Especificaciones técnicas del proyecto
  \item  \textcolor{rojoOscuroS}{$\blacksquare$} Planificación
  \item  \textcolor{naranjaS}{$\blacksquare$} Análisis y diseño
  \item  \textcolor{verdeS}{$\blacksquare$} Construcción
  \item  \textcolor{pistachoS}{$\blacksquare$} Documentación
\end{itemize}



\begin{figure}[htb]
\centering
\includegraphics[width=1\textwidth]{./imagenes/gantt}
\caption{Diagrama de Gantt
} \label{fig:gantt}
\end{figure}


\title{Recursos humanos}
\section{Recursos humanos}

Gracias a que es un proyecto de software/hardware libre, todo aquel que
desee particiar en el proyecto puede hacerlo realizando aportaciones en los
\href{https://github.com/iblancasa/ArduBand}{repositorios del proyecto}.\\

Para el desarrollo del proyecto se ha contado con solo un miembro, el autor de este trabajo.\\

\title{Recursos reutilizables}
\section{Recursos técnicos reutilizables}

Estos recursos hacen referencia a anteriores desarrollos hardware o software creados
por otros desarrolladores o empresas.\\

En el capítulo de implementación se explicará con más detalle por qué se
eligen los elementos que se van a describir en los dos siguientes apartados.\\


\subsection{Recursos software reutilizables}
\title{Recursos software reutilizable}

En la introducción se hacía hincapié en la necesidad de crear una aplicación
Android para ayudar al director a establecer el tempo deseado. Por tanto,
serán recursos a tener en cuenta durante el desarrollo todas las herramientas
que implementa el propio SDK de Android y aquellas bibliotecas con licencia
libre que se encuentran en Internet (ya sean para añadir funcionalidad o elementos
gráficos).\\


También se va a desarrollar un dispositivo físico que necesitará un software para estar
controlado. Todas las bibliotecas del controlador con licencia libre, también estarán
disponibles para su uso (en la sección de implementacion, veremos que se utilizará
Arduino -el cual tiene muchas bibliotecas disponibles- y XBee -existiendo para este
sistema de comunicación Wireless algunas bibliotecas desarrolladas por la comunidad-).\\



\subsection{Recursos hardware reutilizables}
\title{Recursos hardware reutilizable}

Como se acaba de comentar, se utilizará la plataforma Arduino. Además de poder comprar
placas ya fabricadas, existe la posibilidad de crear una propia (es hardware libre y
los planos y software se encuentran disponibles en GitHub \cite{arduinoRepo}).\\

Se utilizará también XBee, de la compañía Digi \cite{xbeedatasheet}.\\

Por otro lado, se puede considerar el dispositivo móvil Android como un elemento a
tener en cuenta en esta sección (ya que es el elemento que hará de interfaz con el usuario,
eliminando la necesidad de crear un dispositivo físico para que el director interactúe con
el sistema).\\


\section{Presupuesto}
\title{Presupuesto}

En esta sección se va a proceder a estimar los costes del proyecto.\\

\subsection{Licencias software y hardware}
\title{Licencias software y hardware}

Gracias a la utilización de software y hardware libre, no es necesario el pago de licencias.\\

El único problema que podría plantearse viene por parte de la empresa Digi al utilizar
el dispositivo XBee pero, debido a que este dispositivo está pensado para crear otros
nuevos, no existe problema en cuanto a su redistribución. X-CTU, el software que se va a
utilizar para la programación de los módulos no tiene licencia libre y se encuentra protegido
por derechos de autor, aunque se distribuye de manera gratuita \cite{licenciaXCTU}.\\


\subsection{Material}
\title{Material}

A continuación se listarán los materiales necesarios y su costo aproximado.
  \begin{description}
    \item [Arduino(*)]\hfill \\
      \begin{itemize}
        \item {Función: controlar los elementos hardware}
        \item {Precio aproximado: 7\euro}
      \end{itemize}
    \item [XBee]\hfill \\
      \begin{itemize}
        \item {Función: comunicación con el resto de dispositivos}
        \item {Precio aproximado: 25\euro}
      \end{itemize}
      \item [Bluetooth]\hfill \\
        \begin{itemize}
          \item {Función: comunicación entre el dispositivo ArduBand y un dispositivo móvil}
          \item {Precio aproximado: 6\euro}
        \end{itemize}
      \item [XBee Adapter/shield]\hfill \\
        \begin{itemize}
          \item {Función: permitir la conexión de XBee con Arduino}
          \item {Precio aproximado: 10\euro}
        \end{itemize}
      \item [Micromotor]\hfill \\
        \begin{itemize}
          \item {Función: actuador. Informa al músico del pulso}
          \item {Precio aproximado: 1,5\euro}
        \end{itemize}
      \item [Carcasa impresa]\hfill \\
          \begin{itemize}
            \item {Función: guardar los componentes de posibles golpes}
            \item {Precio aproximado: 1\euro}
        \end{itemize}
      \item [Varios]\hfill \\
        \begin{itemize}
          \item {Función: cableado, soldadura...}
          \item {Precio aproximado: 2\euro}
        \end{itemize}
  \end{description}


  \begin{table}[h]
  \centering
  \label{table:costodirector}
  \begin{tabular}{ll}
  \hline
  \rowcolor[HTML]{9698ED}
  {\bf Material}      & {\bf Precio (\euro)} \\ \hline
  Arduino*            & 7                \\
  XBee                & 25               \\
  Bluetooth           & 6                \\
  XBee Adapter/shield & 10               \\
  Carcasa             & 1                \\
  Varios              & 2                \\
  \rowcolor[HTML]{CBCEFB}
  {\bf Total}         & 51              \\ \hline
  \end{tabular}
      \caption{Coste de un dispositivo director}
  \end{table}



  \begin{table}[h]
  \centering
  \label{table:costomusico}
  \begin{tabular}{ll}
  \hline
  \rowcolor[HTML]{9698ED}
  {\bf Material}      & {\bf Precio (\euro)} \\ \hline
  Arduino*            & 7                \\
  XBee                & 25               \\
  Micromotor vibrador & 1,50             \\
  XBee Adapter/shield & 10               \\
  Carcasa             & 1                \\
  Varios              & 2                \\
  \rowcolor[HTML]{CBCEFB}
  {\bf Total}         & 46,5               \\ \hline
  \end{tabular}
    \caption{Coste de un dispositivo músico}
  \end{table}


(*)Como Arduino es un proyecto de hardware libre, muchas empresas han decidido
dedicarse a crear sus propias placas compatibles con las originales pero a un
precio mucho menor. En este presupuesto se asume que se están utilizando estas
placas derivadas.\\

A este material necesario se podría añadir las herramientas de las que se precisa
para el desarrollo, es decir, los costes no recurrentes en ingeniería.\\

\begin{description}
  \item [Dispositivo Android]\hfill \\
    \begin{itemize}
      \item {Función: desarrollar y depurar la aplicación Android}
      \item {Precio aproximado: 150\euro}
    \end{itemize}
  \item [Dispositivo ``Android Wear"]\hfill \\
    \begin{itemize}
      \item {Función: desarrollar y depurar la aplicación Android Wear}
      \item {Precio aproximado: 100\euro}
    \end{itemize}
  \item [Soldador]\hfill \\
    \begin{itemize}
      \item {Función: soldar los elementos que sean necesarios}
      \item {Precio aproximado: 10\euro}
    \end{itemize}
    \item [Ordenador]\hfill \\
      \begin{itemize}
        \item {Función: programar el software, documentar el código, buscar información...}
        \item {Precio aproximado: 450\euro}
      \end{itemize}
\end{description}

\begin{table}[h]
\centering
\begin{tabular}{ll}
\hline
\rowcolor[HTML]{9698ED}
{\bf Herramienta}        & {\bf Precio (\euro)} \\ \hline
Dispositivo Android      & 150                  \\
Dispositivo Android Wear & 100                  \\
Soldador                 & 10                   \\
Ordenador                & 450                  \\
\rowcolor[HTML]{CBCEFB}
{\bf Total}              & 710                  \\ \hline
\end{tabular}
\caption{Coste de las herramientas}
\label{costedelasherramientas}
\end{table}


En el caso del dispositivo Android se ha elegido (para el presupuesto) ``Motorola Moto G" por tener
un precio basante ajustado y soportar la última versión de Android (cuando se escribe este
trabajo, la última versión es Android 5). Para dispositivo ``Android Wear" (también para el presupuesto)
se ha elegido ``LG G Watch" (por ser el dispositivo ``Android Wear" más barato del momento
y que, con las funcionalidades de las que dispone, permite llevar a cabo el desarrollo). De los dispositivos
Android se puede prescindir al poder emularlos en un ordenador. También se podría sacar
de este presupuesto el ordenador y el soldador (ya que son herramientas básicas y se asume
que el desarrollador debe de tenerlas antes de empezar con el proyecto).\\

\subsection{Recursos humanos}
\title{Recursos humanos}

Como se ha comentado anteriormente en la sección dedicada a hablar de los recursos humanos disponibles,
al ser un desarrollo libre, cualquiera que lo desee puede participar. Sin embargo, para llevar a cabo la
construcción del sistema, es necesario que haya gente trabajando en él. Los cálculos
se realizarán partiendo de que hay un trabajador con el título de Grado en Ingeniería Informática.\\

Retomando los datos de la estimacióm temporal, tenemos que el número total de horas es
de 215 horas. Pero de esas horas, no podemos tomar el 100\% fomo facturable (entrevistas,
búsqueda de información, formación y similares no se pueden introducir en facturación),
lo que nos lleva a tener que hacer una nueva estimación sobre el número de horas facturables
(de las anteriormente descritas).\\


\begin{itemize}
  \item Especificaciones del proyecto
  \begin{itemize}
    \item{Tiempo facturable estimado: \textbf{6 horas}}
  \end{itemize}
\end{itemize}

\begin{itemize}
  \item Planificación
  \begin{itemize}
    \item{Tiempo facturable estimado: \textbf{0 horas}}
  \end{itemize}
\end{itemize}

\begin{itemize}
  \item Análisis y diseño
  \begin{itemize}
    \item{Analizar los requisitos del proyecto}
    \item{Creación de diagramas}
    \item{Diseño de arquitectura}
    \item{Tiempo facturable estimado: \textbf{40 horas}}
  \end{itemize}
\end{itemize}


\begin{itemize}
  \item Construcción
  \begin{itemize}
    \item{Tiempo facturable estimado: \textbf{75 horas}}
  \end{itemize}
\end{itemize}

\begin{itemize}
  \item Documentación
  \begin{itemize}
    \item{Tiempo facturable estimado: \textbf{35 horas}}
  \end{itemize}
\end{itemize}


\begin{table}[h]
\centering
\begin{tabular}{ll}
\hline
\rowcolor[HTML]{9698ED}
{\bf Actividad}               & {\bf Tiempo (horas)} \\ \hline
Especificaciones del proyecto & 6                    \\
Planificación                 & 0                    \\
Análisis y diseño             & 40                   \\
Construcción                  & 75                   \\
Documentación                 & 35                   \\
\rowcolor[HTML]{CBCEFB}
{\bf Total}                   & 156                  \\ \hline
\end{tabular}
\caption{Número de horas facturables en el proyecto}
\label{table:horasfacturables}
\end{table}


Debido a que, por Ley, está prohibido que los Colegios Profesionales dispongan e informen
sobre baremos orientativos de honorarios, solo podemos obtener el sueldo medio de un Graduado
en Ingeniería Informática a través de estudios. El Consejo General de Colegios Oficiales de
Ingeniería en Informática elaboró en abril de 2015 un estudio \cite{estudioSalario} en el que
podemos ver (gráfica 34, en la página 39) que el sueldo oscila entre los 12.000\euro\ y
30.000\euro\ (la mayoría de los trabajadores se encuentran en este rango). Tomaremos que
un trabajador de estas características cobra 16.000\euro\ al año (para suponer un caso
más caro intermedio dentro del rango más común).\\

Teniendo esta cifra, puede parecer directo obtener cuánto cobra un trabajador a la hora. Sin
embargo, no es así. En España, la jornada laboral es de un máximo de 40 horas a la semana (5 días
multiplicado por 8 horas diarias) y el año tiene en torno a 52 semanas, lo que hace un total de 2.080 horas.
Pero a estas horas hay que restarle festivos, vacaciones, asuntos propios, enfermedad... en media
son unos 38 días (21 de vacaciones, 14 de días festivos y 3 que se corresponden con asuntos propios y/o
enfermedad), es decir, 304 horas. Las horas de trabajo se reducen a 1.776 horas por año. Por tanto, ya
podemos calcular el gasto que nos va a suponer en horas de ingeniería (después habría que sumar
las cantidades relacionadas con Seguridad Social, seguros y similares).\\


\[
  30.000 \div 1.776 = 9 euros/hora
\]
\[
  9 \times 156 = 1.404 euros
\]

\subsection{Otros gastos}
\title{Otros gastos}
Teniendo en cuenta que se va a desarrollar una aplicación ``Android" y otra ``Android Wear",
será necesario pagar la cuota para obtener licencia de desarrollo (25USD, unos 22,5\euro) \cite{desarrollaAndroid}.\\


\subsection{Gasto total}
\title{Gasto total}
Deberemos diferenciar entre varios costos.

  \begin{itemize}
    \item Material para dispositivos
      \begin{itemize}
        \item Dispositivo director: 51\euro/dispositivo
        \item Dispositivo músico: 46.5\euro/dispositivo
      \end{itemize}
    \item Herramientas de desarrollo: 710\euro
    \item Recursos humanos: 1.404\euro
    \item Licencia de desarrollo Android: 22,5\euro
  \end{itemize}

Teniendo en cuenta que para comprobar que el sistema funciona, es decir, que hay
coordinación, hay que desarrollar un dispositivo director y,  al menos, dos de músico,
el precio del material es de 144.5\euro.\\

El precio final del desarrollo sería de \textbf{2.281\euro}. De esta cifra podríamos
descontar el precio de las herramientas de desarrollo (por los motivos antes comentados).
También, al comprar al por mayor, el precio de los materiales se vería considerablemente
reducido.\\

En cuanto a los costos derivados de los recursos humanos, cuanto más dispositivos se produzcan,
menor será el costo. Es decir, la inversión se verá ``más repartida". También habrá que tener en
cuenta que, al producir nuevos dispositivos, es tiempo que hay que pagar a la mano de obra
(pero será tiempo ensamblando el producto, por lo que no se estará gastando tiempo en análisis,
diseño... -salvo que se desee añadir una nueva funcionalidad-). El coste estático (sin contar herramientas)
sería de 1426\euro (los costes de cada dispositivo son gastos dinámicos).\\

Vamos a hacer un cálculo aproximado de los gastos que quedarían sin cubrir tras vender algunos dispositivos:

\begin{gather*}
  1426 + (51 * \textup{Directores} +46.5 \times \textup{Músicos}) = \textup{Gasto material} \\ \\
  \textup{Gasto material} - ((\textup{Beneficio/dispositivo} \times \textup{Total dispositivos})+(51 * \textup{Directores}\\
  + 46.5 \times \textup{Músicos})) = \textup{Coste final}
\end{gather*}

Supongamos que a cada dispositivo se le saca un beneficio de 10\euro: si producimos 100
(una sola banda donde 1 es director y 99 músicos):

\[
  1426 + (51+46.5 \times 99) = 6080.5  \textup{\euro}
\]
\[
  6080.5 - ((10 \times 100)+4655) = 426  \textup{\euro}
\]

Nos seguirían faltando por recuperar 426\euro. Si en vez de ser una banda, son 10
(con 1 director y 99 músicos cada una).\\


\[
  1426 + (51*10+46.5*99*10) = 47971  \textup{\euro}
\]
\[
  47971 - (((10*100)*10+46550) = -8574  \textup{\euro}
\]

Que en este último caso la cifra sea negativa indica ganancia (ya que lo que se están
calculando son los gastos). Como se ha apuntado unos párrafos antes, esta cifra no es
del todo real, ya que habría que incluir la mano de obra para desarrollar cada dispositivo
(aún así, en el segundo caso se seguirían obteniendo beneficios ya que un mismo trabajador
podría construir muchos dispositivos a lo largo de una jornada laboral).\\
